\label{sec:introduction}
Optical systems that focus light in a single point have substantial relevance in many engineering fields such as biomedical, photography, lithography and telecommunication \cite{machikhin2015double,bn2022application,brunner1997impact,bjorkqvist2019additive}. By concentrating light into a single point it is possible to form sharp images, print precise patterns in semiconductors \footnote{A foundational manufacturing technology for chips} and direct with high precision radio waves. Thus, the optimization of optical systems with precise focal abilities can yield huge benefits in optics related industries. One example of such system is the Luneburg lens, this is a spherical gradient index (GRIN) distribution that yields analytical ray tracing solutions without aberration in a point of the surface of the lens. \\ 

Although Luneburg lenses are aberration free, they have practical downsides. The spherical shape can be difficult to manufacture when considering wide fields of view leading to dimensional limitations \footnote{A study has shown that if the image resolution of compound  eyes increases to the same level as the human’s single aperture eye, the radius of the overall lens would be at least 1 m \cite{kirschfeld1976resolution}}. In addition, the lens has a continuous distribution, this means that the surface of the lens must have a refractive index equal to its surroundings - another huge manufacturing challenge. In order to overcome such practical downsides, it is possible to conceive the idea of super-lens composed of smaller, individual spherical lenses with distinct GRIN distributions that can concentrate a wide field of view into an aberration-free point. The design presented in the paper consists of two GRIN distributions: the Luneburg lens and a variation of the Gutman lens.\\

When designing optical systems, it is convenient to use ray-tracing to monitor the behavior of light inside the system. In 2021 Jesus Gomez et. al published a paper that puts forward a ray-tracing methods using Fermat's invariant \cite{GomezCorrea21}. Such paper discussed how the Lagrangian formulation of optics gives rise to Fermat's constant which in turn can be used to formulate a very elegant ray-tracing algorithm that unlike traditional methods; can be generalized to many systems. The algorithm consists on calculating Fermat´s invariant $k$ at the beginning of the propagation and then using it to calculate the angle of diffraction at any point along the trajectory. This describes how the light-beam bends inside the GRIN media. Both its simplicity and accuracy makes it particularly convenient when designing optical systems. Biomimicry is also favorable starting point for engineering designs. By exploiting billions of years of evolution many fields of engineering have flourished. Thus we now proceed to present our design inspired by nature´s eyes. \\